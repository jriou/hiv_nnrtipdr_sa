\documentclass{article}


\usepackage{arxiv}

\usepackage[utf8]{inputenc} % allow utf-8 input
\usepackage[T1]{fontenc}    % use 8-bit T1 fonts
\usepackage{hyperref}       % hyperlinks
\usepackage{url}            % simple URL typesetting
\usepackage{booktabs}       % professional-quality tables
\usepackage{amsfonts}       % blackboard math symbols
\usepackage{nicefrac}       % compact symbols for 1/2, etc.
\usepackage{microtype}      % microtypography
\usepackage{lipsum}
\usepackage{tikz}
\usepackage{dsfont}
\usepackage{amsmath}
\usepackage{array}
\usepackage{todonotes}
\usepackage{float}
\usepackage{rotating}
\usepackage[toc,page]{appendix} %appendix
\usepackage{authblk}
\usepackage{enumitem}
%\setlist{nolistsep}
\usepackage{siunitx}

\definecolor{maroon}{RGB}{176, 48, 96}
\definecolor{orange2}{RGB}{238, 118, 0}
\newcommand{\colindic}[1]{\textcolor{maroon}{#1}}
\newcommand{\colsurvey}[1]{\textcolor{orange2}{#1}}


\title{Dynamics and drivers of HIV-1 drug resistance to non-nucleoside reverse-transcriptase inhibitors in nine African countries}


\author[a,*]{Julien Riou}
\author[a]{Carole Dupont}
\author[b]{Silvia Bertagnolio}
\author[c,d]{Ravindra K. Gupta}
\author[e,f]{Roger D.~Kouyos}
\author[a,g]{Matthias Egger}
\author[a]{Christian L.~Althaus}
\affil[a]{{\small Institute of Social and Preventive Medicine (ISPM), University of Bern, Bern, Switzerland}}
\affil[b] {{\small HIV/Hepatitis/STI Department, World Health Organization, Geneva, Switzerland}}
\affil[c] {{\small Department of Infection, University College London, London, UK}}
\affil[d] {{\small Africa Health Research Institute, Durban, South Africa}}
\affil[e] {{\small Division of Infectious Diseases and Hospital Epidemiology, University Hospital Zurich}}
\affil[f] {{\small Institute of Medical Virology, University of Zurich, Zurich, Switzerland}}
\affil[g] {{\small Centre for Infectious Disease Epidemiology and Research (CIDER), University of Cape Town, South Africa}}
\affil[*] {{\small Corresponding  author (\texttt{julien.riou@ispm.unibe.ch})}}


\begin{document}
\maketitle

\begin{abstract}
	
	\textbf{Background.} The rise of HIV-1 drug resistance to non-nucleoside reverse-transcriptase inhibitors (NNRTI) is a major problem in countries of southern Africa. Understanding the dynamics and drivers of NNRTI resistance at the country level is of critical importance for planning future antiretroviral therapy (ART) programs. 
	
	\textbf{Methods.} We collected survey data on pretreatment drug resistance (PDR) to NNRTIs in nine countries of southern Africa from 2000 to 2018. We fitted a dynamic transmission model to key indicators of the local HIV-1 epidemics (HIV-1 prevalence, ART coverage  and mortality) and to survey data about NNRTI PDR using a Bayesian hierarchical framework. We estimated two country-level indicators: the proportion of NNRTI PDR that cannot be attributed to ART programmes and the vulnerability to NNRTI PDR within ART programmes. We explored associations between vulnerability to NNRTI PDR and country-level covariates.
		
	\textbf{Findings.} The model reliably described the dynamics of HIV-1 and the dynamics of NNRTI PDR in each country. Predicted levels of NNRTI PDR in 2018 ranged between 3.3\% (95\% credible interval 1.9 to 7.1\%) in Mozambique and 25.3\% (17.9 to 33.8\%) in Eswatini. The main determinant of high NNRTI PDR were the conjunction of high ART coverage and high vulnerability to NNRTI PDR within ART programmes. Heterogeneity in the vulnerability to NNRTI resistance was associated with features of the healthcare financing system at the national level.
	
	\textbf{Interpretation.} Between-country comparison shows that NNRTI PDR can be controlled despite high levels of ART coverage, as in Botswana, Lesotho, Mozambique and Zambia, likely because of better adherence, patient management procedures and quality in HIV care service delivery.
	
	\textbf{Funding.} National Institute of Allergy and Infectious Diseases (grant 5U01-AI069924-05) and Swiss National Science Foundation (grant 174281).



\end{abstract}

% keywords can be removed
\keywords{HIV-1 drug resistance \and antiretroviral therapy \and non-nucleoside reverse-transcriptase inhibitor \and mathematical model \and southern Africa}

\clearpage

%\section*{Research in context}
%
%\subsection*{Evidence before this study}
%
%The roll-out of ART in sub-Saharan Africa has been based on non-nucleoside reverse-transcriptase inhibitors (NNRTIs) and accompanied by increasing levels of pretreatment HIV drug resistance (PDR) to NNRTIs; however, the drivers of resistance at country level are not yet well understood. We searched PubMed for systematic reviews and original research articles published up to January 16th, 2020. We combined terms for “HIV-1 drug resistance”, “NNRTI” and “Africa”. Repeated surveys and systematic reviews have documented a continuous increase of the prevalence of NNRTI PDR reaching more than 10\% in some countries of southern Africa. More detailed country-level estimates of the emergence of  NNRTI PDR have been hampered by the limited number of surveys in some countries. 
%
%\subsection*{Added value of this study}
%
%To our knowledge this is the first modelling study comparing the emergence of NNRTI PDR  across countries of southern Africa. The hierarchical structure of the model allowed estimation of time trends of NNRTI PDR even in countries where few surveys were available. The model predicted highly heterogeneous levels of NNRTI PDR in 2018, ranging from 3.3\% in Mozambique to 22.5\% in South Africa and 25.3\% in Eswatini. The levels of ART coverage and the vulnerability to NNRTI PDR in ART access programmes were the main determinants of higher levels of NNRTI PDR in 2018. Better control of  NNRTI PDR in a country was associated with higher levels of international donor funding and characteristics linked to smaller healthcare systems.
%
%\subsection*{Implications of all the available evidence}
%
%The rapid and massive roll-out of ART in some southern African countries may have overburdened health systems, negatively impacting the quality of care and resulting in high levels of NNRTI PDR. However, the rise of NNRTI PDR could be controlled in some countries despite high levels of ART coverage. Concerns about drug resistance should not lead to restrictions on drug availability, but rather promote the implementation of policies aimed at increasing patients’ adherence and preventing treatment failure due to NNRTI resistance.

\section{Introduction}

The rising prevalence of HIV-1 drug resistance is threatening the success of antiretroviral therapy (ART) programmes in southern Africa.
This region has the highest burden of HIV-1 infection worldwide, accounting for about 30\% of persons living with HIV-1 (PLHIV) globally \cite{wang2016estimates}.
Since the early 2000s, first-line ART consisting of one non-nucleoside reverse-transcriptase inhibitor (NNRTI) and two nucleoside reverse-transcriptase inhibitors (NRTI) has been introduced through national access programmes \cite{whoreco}.
As of 2018, between 56 and 92\%  of adults living with HIV-1 in the region were treated with ART \cite{unaidsreport2019}. 
The roll-out of NNRTI-based ART has been followed by an increase of pretreatment HIV-1 drug resistance (PDR) to NNRTIs, which is associated with poor virological outcomes \cite{wittkop2011effect,hamers2012effect,avila2016pretreatment}, especially as access to resistance testing and switching to second-line ART are rare in this region \cite{haas2015monitoring}.
NNRTI PDR is the most common measure used to monitor the levels of NNRTI resistance in a population, and is defined as the proportion of PLHIV with NNRTI resistance mutations before or at the time of ART initiation. 
While NNRTIs are being replaced by the integrase inhibitor dolutegravir (DTG) in southern Africa, NNRTIs will likely remain an important drug for the management of HIV-1, especially considering the safety warnings in people exposed to DTG \cite{zash2018neural,norwood2017weight}.
In addition, there is a risk of DTG resistance developing \cite{rhee2019systematic}.
A better understanding of the factors that have led to the development of NNRTI resistance will be important in order to control future DTG resistance.

Several elements can influence the dynamics of NNRTI PDR. 
First, {\em de novo} NNRTI resistance mutations can emerge in HIV-1 infected patients on first-line, NNRTI-based ART who do not suppress viral replication \cite{clavel2004hiv,kepler1998drug,zur2011population}.
Virological failure is precipitated by suboptimal quality of HIV service delivery within ART programmes (including poor adherence support, poor patient retention on antiretroviral therapy or unreliable drug supply), or by off-programme, inappropriate antiretroviral use or single drug regimens used for the prevention of mother-to-child transmission (PMTCT) \cite{hamers2018hiv}.
Second, the high number of PLHIV who are on first-line ART induces a selective advantage of drug-resistant HIV-1 strains over drug-sensitive strains at the population level as transmission is increased in situations of virological failure. 
Switching to second-line ART reduces the risk of virological failure and severe outcome in individuals with NNRTI-resistant strains of HIV-1 \cite{gsponer2012causal,ramadhani2016effect}. 
The rate at which resistant strains replace the wild-type strains in populations of PLHIV is typically proportional to the selective pressure induced by ART programmes, i.e. the treatment rate of HIV-1 infected individuals \cite{bonhoeffer1997virus,fingerhuth2016antibiotic}.
Third, the different dynamics of the HIV-1 epidemics in each country might also influence the exposure of the susceptible population to infection with NNRTI-resistant strains, and ultimately the evolution of NNRTI PDR. 

Dynamic transmission models provide a key tool to better understand the underlying mechanisms that shape the local epidemics and the rise of NNRTI PDR \cite{hauser2019bridging}.
We developed a dynamic transmission model of HIV-1 that describes the local dynamics of HIV-1 transmission and ART roll-out in nine countries of southern Africa. 
We fitted the model to data on the local HIV-1 epidemic and from surveys of NNRTI PDR in each country using a hierarchical Bayesian framework. 
We estimated and compared the trajectories of NNRTI PDR in each country between 2000 and 2018. 
We propose two indicators that describe and compare different aspects of the growth of NNRTI PDR in each country. 
First, we quantified what proportion of NNRTI PDR can be attributed to ART programmes and what proportion can be attributed to off-programme antiretroviral use. 
In each country, we estimated  an indicator of the vulnerability to NNRTI PDR within ART programmes, quantifying the level of control of NNRTI PDR that was achieved after adjusting for the local dynamics of HIV-1 transmission and ART roll-out. 
We then explored associations between  vulnerability to NNRTI PDR and the characteristics of health systems. 

\section{Methods}

\subsection{Data}

We used data from a systematic review of surveys of PDR in HIV-1-infected adults (aged >15 years) in low- and middle income countries \cite{gupta2018hiv}.
The review included publications and abstracts published from January 1st, 2001 to December 31st, 2016 as well as unpublished data from surveys that were supported by the World Health Organization (WHO). We updated the review for the period from January 1st, 2017 to July 31st, 2019 in PubMed, Embase and WHO reports using the same search strategy (appendix section 1). We screened studies for eligibility based on title and abstract. To be eligible the studies had to use original survey data about the proportion of NNRTI resistance in a sample of adults (>15 years) before or at the time of ART initiation, and those data had to be collected between 2000 and 2018 in nine countries of southern Africa (Botswana, Eswatini, Lesotho, Malawi, Mozambique, Namibia, South Africa, Zambia and Zimbabwe). Two independent reviewers (JR and CD) assessed the full manuscripts of potentially eligible studies. 
We used data from 56 surveys of NNRTI PDR from southern Africa that were identified in the original review. The updated search yielded 1,039 items, leading to eight additional eligible studies. In total, we thus included data from 64 surveys of NNRTI PDR in nine countries, for a total of 14,567 individuals. For each country, we obtained yearly estimates regarding four key indicators of the HIV-1 epidemic for each year from 2000 to 2018: (A) the number of adults living with HIV, (B) the number of adults living with HIV-1 receiving ART, (C) the number of AIDS-related deaths among adults and (D) the size of the adult population. Datasets (A-C) were retrieved from UNAIDS reports and dataset (D) from World Bank data \cite{unaidsreport2019,worldbank}.
Data used in this analysis were not raw data from individual patients, but rather aggregates at the country level as reported in WHO published reports and other publications available in the literature.

\subsection{Model}

We developed a dynamic transmission model linking data on the four key indicators of the local HIV-1 epidemic (HIV-1 prevalence, ART coverage, mortality and adult population size) to survey data on NNRTI PDR in each country. 
The model consists of a system of ordinary differential equations with six compartments (figure \ref{fig:odemodel}) and considers the following features:

\begin{itemize}
\item \textbf{HIV-1 transmission}.
The force of infection of drug-sensitive and drug-resistant HIV-1 strains depends on the transmission rate $\beta$. We assumed that only individuals without treatment (compartments $I$ and $J$) or who fail first-line treatment (compartment $U$) can transmit HIV-1 \cite{supervie2014heterosexual}, and that there is no fitness cost associated with the transmission of drug-resistant strains \cite{little2008persistence}.

\item \textbf{First-line NNRTI-based ART}.
Untreated individuals initiate first-line ART at a time-dependent rate $\tau f(t,\nu,\xi)$, where the maximal treatment rate $\tau$ is scaled by a sigmoid function $f(t,\nu,\xi)$ to reflect the progressive roll-out of ART.

\item \textbf{Emergence of \emph{de novo} NNRTI resistance emergence}. 
HIV-1 infected adults on first-line ART can acquire \emph{de novo} NNRTI resistance at rate $\omega$ (DNR).

\item \textbf{Initial levels of NNRTI PDR.} We defined the starting date of the model in 1999, before the implementation of ART programmes. We describe the level of NNRTI PDR at that date by an intercept parameter $\iota$. Similar to the intercept in a general linear model, this parameter describes the proportion of NNRTI PDR that cannot be explained by the roll-out of ART within ART programmes. 

\item \textbf{Second-line NNRTI-free ART}. 
Individuals who fail first-line ART due to NNRTI resistance may be switched to second-line NNRTI-free ART regimens at rate $\kappa$.

\item \textbf{Demography}.
New individuals enter the adult population (>15 years) at rate $\eta$. 
All individuals can die from background mortality at rate $\mu$ which we fix to the inverse of the life expectancy in adults in each country. HIV-1 infected individuals without treatment (compartments $I$ and $J$) or who fail first-line ART (compartment $U$) die from AIDS-related mortality at rate $\delta$.
\end{itemize}

\begin{figure}[t]
	\centering
	\includegraphics[width=1\linewidth]{../figures/fig1c.pdf}
	
	\caption{Diagram of the dynamic model of HIV-1 transmission, ART roll-out and resistance emergence. The adult population is split into six compartments: susceptible to HIV-1 ($S$); infected with a drug-sensitive strain and either untreated ($I$) or treated with first-line ART ($T$); infected with a drug-resistant strain and either untreated ($J$), treated with first-line ART ($U$) or treated with second-line ART ($V$). The model is controlled by ten parameters: transmission rate $\beta$; maximal treatment rate $\tau$ scaled by a sigmoid function $f(t,\nu,\xi)$ (controlled by two parameters for shift $\nu$ and slope $\xi$); rate at which \emph{de novo} NNRTI resistance emerges during treatment ($\omega$); rate of switching to second-line ART ($\kappa$); rate of AIDS-related mortality ($\delta$); rate of background mortality ($\mu$); rate of population growth ($\eta$); and the initial proportion of NNRTI resistance $\iota$ (not shown). Model outputs are shown on the right, with colour code corresponding to figures 2 and 3.}
	\label{fig:odemodel}
\end{figure}



%
%\begin{table}[h]
%	\centering
%	\caption{Parameters of the HIV-1 transmission model.}
%	\label{table:priors}
%	\begin{tabular}{lp{9.5cm}p{2.9cm}l}
%		\hline \\[-.8em]
%		Parameter & Interpretation & Prior & Unit \\
%		\hline \\[-.7em]
%		\multicolumn{4}{l}{{\em Country-level parameters}} \\[.2em]
%		\hspace{.8em}$\beta_j$ & Transmission rate & $\text{Expon}(5)$ & y$^{-1}$\\ 
%		\hspace{.8em}$\tau_j$ & Maximum rate of first-line ART initiation & $\text{Expon}(5)$ & $y^{-1}$\\ 
%		\hspace{.8em}$\xi_j$ & Slope of the sigmoïd function $f(t)$  & $\text{Beta}(2,2)+0.5$ & -\\ 
%		\hspace{.8em}$\nu_j$ & Shift of the sigmoïd function $f(t)$  & $\text{Expon}(0.5)$& y  \\ 
%		\hspace{.8em}$\omega_j$ & Rate of treatment failure with NNRTI resistance (TFNR) & $\text{Gamma}(\mu_{\omega},\sigma_{\omega})$ & y$^{-1}$\\ 
%		\hspace{.8em}$\kappa_j$ & Rate of switching to second-line ART &$\text{Expon}(5)$& y$^{-1}$ \\ 
%		\hspace{.8em}$\delta_j$ & AIDS-related mortality rate &$\text{Expon}(5)$& y$^{-1}$ \\ 
%		\hspace{.8em}$\eta_j$ & Adult population growth rate & $\text{Expon}(5)$ & y$^{-1}$\\ 
%		\hspace{.8em}$\iota_j$ & Initial proportion of NNRTI resistance in 2000 & $\text{inv.logit}\left(\mathcal{N}(\mu_{\iota},\sigma_{\iota})\right)$ & y$^{-1}$\\
%		\hspace{.8em}$\mu_j$ & Adult population background mortality rate$^{\dagger}$ & - & y$^{-1}$\\[.2em]
%		\hline \\[-.7em]
%		\multicolumn{4}{l}{{\em Hyperparameters}} \\[.2em]
%		\hspace{.8em}$\mu_{\omega}$ & Location hyperparameter for the distribution of $\omega_j$ & $\text{Expon}(5)$ & y$^{-1}$\\ 
%		\hspace{.8em}$\sigma_{\omega}$ &  Scale hyperparameter for the distribution of  $\omega_j$ & $\text{Expon}(20)$ & y$^{-1}$\\ 
%		\hspace{.8em}$\mu_{\iota}$ & Location hyperparameter for the distribution of $\iota_j$ & $\text{logit}\left(\text{Beta}(1,9)\right)$ & y$^{-1}$\\ 
%		\hspace{.8em}$\sigma_{\iota}$ &  Scale hyperparameter for the distribution of  $\iota_j$ & $\text{Expon}(5)$ & y$^{-1}$\\
%%		\hspace{.8em}$\sigma^a_j,\cdots,\sigma^d_j$ & Scales for normal distributions & $\text{Expon}(1)$& -\\[.2em] 
%		\hline 
%		\multicolumn{4}{l}{{\small$^{\dagger}$Background mortality was fixed to the inverse of adult life expectancy in each country.}} \\
%	\end{tabular} 
%\end{table}

\subsection{Indicators}

Two country-level indicators were derived from these parameters:
\begin{itemize}
	\item First, the \textbf{proportion of NNRTI PDR in 2018 unrelated to ART programmes} was defined as $\iota / PDR_{2018}$, where the numerator is the intercept of NNRTI PDR in the country and the denominator is the model-predicted level of NNRTI PDR in 2018. This indicator quantifies the relative impact of undocumented, off-programme antiretroviral use (e.g., for PMTCT) on NNRTI PDR in a country.
	
	\item Second, the \textbf{vulnerability to NNRTI PDR within ART programmes} was defined as $\omega/\kappa$, i.e. the ratio of the rate of occurrence of NNRTI resistance in adults treated with first-line ART over the rate of switching to second-line ART. It can be interpreted as a standardized proxy measure for the growth of NNRTI PDR in a country as a result of first-line ART initiation within ART programmes. As it is not dependent on the scale and timing of ART roll-out, it can be directly compared across countries. Low values (slow occurrence of NNRTI resistance under first-line ART and rapid switch to second-line ART) and high values (rapid occurrence of NNRTI resistance and slow switch to second-line ART) for this indicator correspond to better or worse control of NNRTI PDR within ART programmes.
\end{itemize}



\subsubsection{Bayesian hierarchical framework}

We implemented the model in the Bayesian statistical software \texttt{Stan} 2.18.2 \cite{carpenter2017stan}.
The ODEs for each country were numerically integrated in parallel, with a starting date in 1999. 
The was a total of nine free model parameters $\{\beta,\tau,\xi,\nu,\omega,\kappa,\delta,\eta,\iota\}$ and one fixed parameter $\mu$ for each country. 
We imposed a hierarchical structure on the parameters related to NNRTI resistance ($\omega$ and $\iota$). The other parameters were independently estimated for each country. We selected weakly-informative prior distributions for all free parameters, and conducted prior predictive checks to ensure that the chosen priors limited the range of explored parameter space to sensible values \cite{gabry2019visualization}. By fitting the full hierarchical model to all available data, we obtained the joint posterior distribution for the nine free model parameters in each country. Further details about the model and the fitting procedure are available in the appendix.

\subsubsection{Socio-economic covariates}

We explored associations between vulnerability to NNRTI PDR within ART programmes and eight country-level variables measuring characteristics of health systems and countries: pregnant women who received NNRTIs for prevention of  mother-to-child transmission (PMTCT, number of pregnant women who received NNRTIs to prevent vertical transmission  as a proportion of HIV-1 adult prevalence), total health expenditure per capita (in \$ purchasing power parity), proportion of international donor funding in total health expenditure, proportion of out-of-pocket health expenditure, gross national income per capita (in \$), number of hospital beds per capita, proportion of rural population, proportion of unemployed. Data were retrieved from the World Bank or UNAIDS \cite{worldbank,unaidsreport2019} and we used the mean over 2000-2018 for this analysis. We computed Spearman's rank correlation coefficients between the estimated indicator of vulnerability to NNRTI PDR and each country-level covariate, as rank-based correlation is not sensitive to extreme values.


\subsection{Role of the funding source}
Study funders had no role in study design, data analysis, interpretation of results, or writing of the report.

\section{Results}

The model accurately described the dynamics of HIV-1 prevalence, ART coverage, AIDS-related mortality and population size in each country for the period 2000-2018 (figure \ref{fig:indic}).
The levels and trends of HIV-1 adult prevalence varied by country, with higher prevalences in Botswana, Eswatini and Lesotho and lower, decreasing prevalence in Malawi, Zambia and Zimbabwe. 
The timing and intensity of the ART roll-out varied by country (figure \ref{fig:misc}A). 
The roll-out occurred first in Botswana and last in Mozambique. The maximum intensity at which patients initiated first-line ART was lowest in Lesotho and highest in Zimbabwe. Model fit to AIDS-related mortality was less accurate, but the magnitudes and overall trends of mortality in each country were well captured by the model.


\begin{table}[ht]
	\centering
	\caption{Country-level estimates of the main aspects of ART roll-out and NNRTI PDR in southern Africa (median posterior and 95\% credible interval). The timing of ART roll-out refers to the date at which 50\% of the maximum treatment rate is reached. The intensity of ART roll-out refers to the maximal treatment rate. Total NNRTI PDR in 2018 is the model-predicted level of NNRTI PDR.}
	\label{tab:post}
	\scalebox{.83}{
		\begin{tabular}{lp{2.6cm}p{2.6cm}p{2.7cm}p{2.2cm}p{2.6cm}p{2.6cm}}
			\hline
			
			Country & Timing of ART roll-out &  Intensity of ART roll-out & Total NNRTI PDR in 2018 & NNRTI PDR unrelated to ART access programmes & Proportion of NNRTI PDR unrelated to ART access programmes & Vulnerability to NNRTI PDR within ART access programmes\\ 
			\hline
			Botswana & 2005 (2004-2006) & 0.13 (0.12-0.15) & 3.8\% (2.5-5.6) & 1.5\% (0.7-2.4) & 41\% (15-74) & 0.01 (0.00-0.12) \\ 
			Eswatini & 2009 (2009-2009) & 0.24 (0.18-0.29) & 25.3\% (17.9-33.8) & 1.5\% (0.5-3.0) & 6\% (2-14) & 0.65 (0.23-9.36) \\ 
			Lesotho & 2007 (2006-2007) & 0.09 (0.083-0.1) & 3.8\% (1.4-9.7) & 2.2\% (0.8-4.7) & 63\% (21-87) & 0.01 (0.00-0.62) \\ 
			Malawi & 2009 (2009-2009) & 0.22 (0.18-0.28) & 8.7\% (4.2-23.9) & 3.1\% (2.1-4.4) & 38\% (11-71) & 0.04 (0.00-0.95) \\ 
			Mozambique & 2009 (2009-2009) & 0.11 (0.095-0.13) & 3.3\% (1.9-7.1) & 2.0\% (1.3-3.0) & 64\% (26-85) & 0.01 (0.00-0.63) \\ 
			Namibia & 2007 (2006-2008) & 0.18 (0.15-0.21) & 24.9\% (17.9-32.3) & 2.1\% (0.7-4.5) & 8\% (3-22) & 0.48 (0.16-10.17) \\ 
			South Africa & 2009 (2009-2009) & 0.13 (0.12-0.14) & 22.5\% (19.1-26.0) & 1.7\% (1.2-2.3) & 8\% (5-11) & 1.19 (0.83-6.98) \\ 
			Zambia & 2008 (2007-2009) & 0.17 (0.14-0.21) & 6.3\% (3.4-13.6) & 3.1\% (2.1-4.4) & 50\% (22-80) & 0.01 (0.00-0.27) \\ 
			Zimbabwe & 2009 (2009-2009) & 0.25 (0.2-0.32) & 13.4\% (6.6-22.6) & 3.9\% (3.1-4.8) & 29\% (15-61) & 0.15 (0.00-2.67) \\
			\hline
	\end{tabular}}
\end{table}



\begin{sidewaysfigure}
	\centering
	\includegraphics[width=\linewidth]{../figures/post_indicators_M6.pdf}
	\caption{Model fit for four key indicators of the HIV-1 epidemic in nine countries of southern Africa from 2000 to 2018: HIV-1 prevalence (as a proportion of the adult population), ART coverage (as a proportion of HIV-1 prevalence), AIDS-related mortality (as a proportion of HIV-1 prevalence) and adult population size (in millions).  Circles represent data, full lines and shaded areas correspond to model predictions (median posterior and 95\% prediction interval).}
	\label{fig:indic}
\end{sidewaysfigure}

Linking the dynamics of HIV-1 transmission, treatment and mortality to survey data on NNRTI PDR, the model estimated the rise of NNRTI PDR in every country (figure \ref{fig:pdr}). 
There was good agreement between the estimated trajectory of NNRTI PDR between 2000 and 2018 and survey data. 
There were only two outliers (defined as survey estimates with confidence intervals that did not overlap with modelled estimates), one from Malawi among persons with acute HIV-1 infection \cite{rutstein2019high} and another one from South Africa among sex workers \cite{coetzee2017hiv}.

Model-predicted levels of NNRTI PDR in 2018 ranged between 3.3\% (95\% credible interval [CrI] 1.9 to 7.1) in Mozambique and 25.3\% (17.9 to 33.8) in Eswatini (table \ref{tab:post}). Large uncertainty intervals reflected the scarcity of recent survey data in Eswatini, Lesotho, Malawi, Namibia and Zambia. The absolute amount of NNRTI PDR in 2018 that cannot be attributed to ART programmes ranged between 1.5\% in Eswatini and Botswana and 3.9\% in Zimbabwe. In relation to the levels of NNRTI PDR in 2018, the proportion of NNRTI PDR unrelated to ART programmes was small in Eswatini, Namibia and the Republic of South Africa (from 6\% [95\%CrI: 2-14] in Eswatini to 8\% [95\%CrI: 3-22] in Namibia). In contrast, NNRTI PDR was most influenced by off-programme antiretroviral use in Zambia, Lesotho and Mozambique (from 50\% [95\%CrI: 22-80] in Zambia to 64\% [95\%CrI: 26-85] in Mozambique).

The magnitude of first-line ART coverage together with high vulnerability to NNRTI PDR within ART programmes were the main drivers of the large increases of NNRTI PDR observed in some countries (figure \ref{fig:misc}C). 
Country-specific estimates of the indicator of vulnerability to NNRTI PDR within ART programmes showed marked differences across countries (figure 4B). 
In Botswana, Lesotho, Mozambique and Zambia, estimates of this indicator were very low, close to 0.01 in all four countries, indicating that the number of patients acquiring {\em de novo} NNRTI resistance during first-line ART was compensated by a rapid switch to second-line ART, thereby preventing the further spread of NNRTI resistance (table \ref{tab:post}).  
Conversely, vulnerability to NNRTI PDR was highest in Eswatini, Namibia and  the Republic of South Africa, ranging from 0.48 (95\% credible interval[CrI]: 0.16-10.17) in Namibia to 1.19 (95\%CrI: 0.83-6.98) in the Republic of South Africa. 
The combination of high vulnerability to NNRTI PDR within ART programmes and high levels of ART coverage was associated with a sharp increase in NNRTI PDR after ART roll-out in these three countries (figure \ref{fig:pdr}).

The variation between countries regarding the vulnerability to NNRTI PDR within ART programmes was partly explained by country-level covariates (table \ref{tab:explo}). Lower estimates of vulnerability to NNRTI PDR were found in countries with higher levels of international donor funding (as a proportion of total health expenditure). Higher estimates of failure were found in countries with higher levels of total health expenditure and hospital beds per 1000 population. We did not find evidence of an association between the level of control of NNRTI PDR within ART programmes and the intensity of PMTCT programmes at the country level.


\begin{figure}[t]
	\includegraphics[width=\linewidth]{../figures/post_pdr_M6.pdf}
	\caption{Model fit for the prevalence of NNRTI pretreatment drug resistance (PDR) in nine countries of southern Africa from 2000 to 2018. Circles represent survey results with 95\% credible intervals. Lines and shaded areas correspond to model predictions (median posterior and 95\% credible interval -- prediction intervals are not practical because of the variation in sample size across surveys).}
	\label{fig:pdr}
\end{figure}


\begin{figure}[t]
	\centering
	\includegraphics[width=.43\linewidth]{../figures/plot_panel_M6b.pdf}
	\caption{(A) Visualizing ART roll-out: change in the rate of ART initiation (per year) between 2000 and 2018 as estimated from the model (median posterior, uncertainty not shown). (B) Estimates of the indicator of vulnerability to NNRTI PDR within ART programmes by country (median posterior). (C) 3-way scatter plot between the posterior samples of: the indicator of vulnerability to NNRTI PDR within ART programmes (y-axis), the proportion of all persons living with HIV that initiated ART over the period 2000-2018 (x-axis) and the predicted levels of NNRTI PDR in 2018 (colour gradient). Circles correspond to 2D-medians by country.}
	\label{fig:misc}
\end{figure}



\begin{table}[ht]
	\centering
	\caption{Exploratory analysis of the country-level drivers of vulnerability to NNRTI PDR within ART access programmes in southern Africa. Negative correlation coefficients translate into an association with a better control of NNRTI PDR.}
	\label{tab:explo}
	\begin{tabular}{lc}
		\hline
		Country-level covariate & Correlation$^{\dagger}$\\ 
		\hline
		External health expenditure (as a proportion of current health expenditure) & -0.45 (-0.80; -0.10) \\ 
		Rural population (as a proportion of total population) & -0.17 (-0.55; 0.18) \\ 
		Out-of-pocket expenditure (as a proportion of current health expenditure) & 0.02 (-0.43; 0.40) \\ 
		PMTCT (as a proportion of total prevalence) & 0.28 (-0.15; 0.63) \\ 
		Unemployment (as a proportion of total labour force) & 0.22 (-0.18; 0.67) \\ 
		Gross national income per capita (in US\$) & 0.28 (-0.08; 0.70) \\ 
		Current health expenditure per capita (in US\$) & 0.53 (0.17; 0.85) \\ 
		Number of hospital beds per 1,000 people &  0.65 (0.25; 0.90) \\ 
		\hline
		\multicolumn{2}{p{13cm}}{\footnotesize{$^{\dagger}$Spearman's rank correlation coefficient between the posterior samples of the indicator of vulnerability to NNRTI PDR within ART access programmes and each covariate (median and 95\% credible interval).}}
	\end{tabular}
\end{table}


\section{Discussion}

This study provides a detailed analysis of the rise of NNRTI PDR in nine countries of southern Africa for the period 2000–2018. We used a dynamic transmission model in a hierarchical Bayesian framework to link the local characteristics of HIV-1 transmission, treatment and mortality at the country level to survey data on NNRTI PDR in adults. Putting together all relevant information available in these countries, our approach allows for a close examination of the dynamics and drivers of NNRTI PDR in this region using two country-specific indicators: the proportion of NNRTI PDR unrelated to ART programmes and the vulnerability to NNRTI PDR within ART programmes.

We found considerable heterogeneity in the way NNRTI PDR grew between 2000 and 2018 across countries of southern Africa. 
The heterogeneity in the rise of NNRTI PDR is driven by country-level differences in both the timing and intensity of ART roll-out and the vulnerability to NNRTI PDR within ART programmes. While the evolutionary pressure exerted by NNRTI-based, first-line ART obtained through ART programmes is a necessary condition for the selection of NNRTI resistance mutations, many other intricate factors greatly influence the levels of NNRTI PDR  found in a country. This is exemplified by the comparison of Botswana and Eswatini, two relatively small countries of the region. ART roll-out started very early on in Botswana, and later but at a higher intensity in Eswatini, leading to a similar number of ART initiations over the period 2000-2018 in proportion to the populations of both countries. In absolute terms, the levels of NNRTI PDR unrelated to ART programmes were estimated to 1.5\% in both countries. Yet the trajectories of NNRTI PDR were very different between these two countries, leading to predictions of NNRTI PDR in 2018 of 25.3\% (17.9 to 33.8) in Eswatini  compared to 3.8\% (2.6 to 5.6) in Botswana. This difference appears to result from a loss of control of NNRTI PDR within the ART programme in Eswatini, as indicated by the values of the vulnerability indicator. The highest estimate of  vulnerability to NNRTI PDR within ART programmes was in the Republic of South Africa, a country where observed levels of NNRTI PDR were very high despite an average intensity of ART roll-out.

The proportion of NNRTI PDR unrelated to ART programmes and the vulnerability to NNRTI PDR within ART programmes can be used to cluster the countries of southern Africa in three groups. The empirical survey data and model predictions show that NNRTI PDR can be controlled and maintained under 10\% in a first group comprised of Botswana, Lesotho, Mozambique and Zambia. This performance is associated with a good control of NNRTI PDR within ART programmes, as indicated by the low values for vulnerability indicator. Such low values prevented the rise of NNRTI PDR  at the population level, even in countries where the roll-out of ART occurred early and massively, such as Botswana (figure \ref{fig:misc}C). In these countries, the observed levels of NNRTI PDR in 2018 are largely related to off-programme antiretroviral use. 
In contrast, NNRTI PDR rose dramatically in a second group including Eswatini, Namibia and the Republic of South Africa. The predicted levels of NNRTI PDR in 2018 ranged between 22.5\% (19.1 to 26.0) in the Republic of South Africa and 25.3\% (17.9 to 33.8) in Eswatini, threatening the success of NNRTI-based ART regimens that remain predominant in the region, although countries are now transitioning to DTG-based ART. Off-programme antiretroviral use had relatively little impact in these countries, and the sharp increase in NNRTI PDR was mostly driven by a combination of a massive roll-out of first-line ART and a high vulnerability to NNRTI PDR within ART programmes. A third group with Malawi and Zimbabwe had intermediate values for each indicator, with NNRTI PDR levels in 2018 around 10\%, related to both ART programmes and off-programme antiretroviral use.

The indicator of vulnerability to NNRTI PDR within ART programmes can be interpreted as a composite indicator of the level of control of NNRTI PDR achieved in a country during ART roll-out, summarizing the global performance of ART programmes \cite{hamers2018hiv}. Drivers such as adherence, patient management and therapeutic education,  quality in HIV care service delivery or previous exposure to ART, are poorly documented \cite{clavel2004hiv,kepler1998drug}. Local policies, such as the sequential modification of ART eligibility criteria  towards earlier treatment initiation, may also drive the emergence of drug resistance \cite{de2018hiv}. Earlier treatment has been linked to higher levels of non-adherence, and the scale-up of treatment programmes may have put a strain on already fragile health systems \cite{nachega2014addressing}. More generally, the level of economic development, the social structure of the population, the funding and organisation of the healthcare system may all interact to drive the emergence of HIV-1 drug resistance. 

Our exploratory analyses of potential drivers show that higher levels of international donor funding is associated with a better control of NNRTI PDR, suggesting that the set of rules and regulations required by international donors (including therapeutic education promoting adherence and guidelines regarding clinical practice) may have a beneficial impact on the control of drug resistance. Conversely, higher levels of total health expenditure and a large number of hospital beds, both associated with a larger healthcare system, were linked to a worse control of NNRTI PDR. This supports the hypothesis that the rapid growth of healthcare systems in some countries may have led to vulnerabilities in patient management. Although PMTCT has long been suspected to be an important driver of NNRTI resistance \cite{hamers2018hiv}, we find no clear association between the intensity of PMTCT programmes by country and estimates of treatment failure associated with NNRTI resistance.

Lessons from the historical growth of NNRTI PDR may have implications for the ongoing roll-out of DTG in southern Africa, even though dolutegravir has a higher genetic barrier to resistance than NNRTIs \cite{llibre2015genetic}. Our results suggest that DTG resistance, once emerged, can be controlled even if the drug is made widely available. This may be achieved by observing strict policies regarding patient education and management, and by improving the general strength and resilience of local healthcare systems. Concerns about drug resistance should not lead to restrictions on DTG availability but rather to more investment in research about interventions aimed at controlling the emergence and spread of resistance. In that regard, implementation science may play an important role on defining and promoting evidence-based practices for routine clinical care.

The main strength of our study is the application of a dynamic transmission model in a hierarchical framework to describe the emergence of NNRTI PDR in different countries. The model describes how the biological and clinical mechanisms of emergence and transmission of NNRTI resistance, together with the dynamics of HIV-1 transmission, treatment and mortality have resulted in disparate trends in NNRTI PDR across southern Africa. Fitted to country-level data using Bayesian inference, the model allowed for the full propagation of uncertainty, and relied on relatively few assumptions beyond model structure.

Our study has several limitations. While the model captured the main trends of the HIV-1 epidemic in each country, its relatively simple structure ignored some dimensions such as age, gender, acute infections and disease progression. In line with our objective of providing a statistically sound assessment of the rise of NNRTI PDR  by country, we focused on mechanisms that could be informed using data available consistently for all countries during the period of interest, and avoided adding features that would require numerical assumptions that would be difficult to support. Furthermore, more dimensions would have quickly expanded the complexity of a model that is already computationally expensive to fit due to its hierarchical nature. These same considerations also led us to ignore an important aspect of surveys measuring NNRTI PDR, the possibility of previous exposure to ART through PMTCT, off-prescription use of ART or undocumented treatment discontinuation. In our model, these mechanisms are grouped together and estimated by considering the amount of NNRTI PDR that is not compatible with the dynamics of ART roll-out. Gupta et al. showed that previous exposure to ART at the point of ART initiation self-reported by patients occurred in 8\% of individuals \cite{gupta2018hiv}. Another limitation concerns the analysis of country-level data, exposing the results to ecological bias and ignoring potential within-country heterogeneity. This choice was dictated by the scale of available data and the fact that HIV-related health policies are implemented at the national level. More data on NNRTI PDR, treatment adherence, previous exposure to ART, virological failure, and ART coverage for first and second-line treatment, collected systematically at the country level or at a lower scale, would be necessary to improve the precision of the estimates.

The roll-out of ART in southern Africa has been followed by increasing levels of NNRTI PDR which continue to increase dramatically. The rise of NNRTI PDR has been highly heterogeneous across countries of the region. Between-country comparison shows that NNRTI PDR can be controlled despite high levels of ART coverage, as in Botswana, Lesotho, Mozambique and Zambia, likely because of better patient management. Our results suggest that the ability to control NNRTI PDR is associated with features of the healthcare system at the national level. A better understanding of the factors that affect the rate at which NNRTI resistance occurs during treatment and the rate at which patients are switched to second-line ART will provide targets for future interventions and help limit the further spread of HIV-1 drug resistance.


\section{Contributors}

JR, ME and CA conceived the study. SB and RG provided data from the previous systematic review. JR and CD updated the systematic review. JR performed the analysis. JR, CD, SB, RG, RK, ME and CA participated in the interpretation of the results and wrote, reviewed and approved the article.

\section{Declaration of interests}

We declare no conflict of interest.

\section{Data sharing}

All data and code are available from \url{https://github.com/jriou/nnrtipdr_sa}.

\section{Acknowledgments}
We thank Leigh Johnson for the helpful discussion.
Model calculations were performed on UBELIX (http://www.id.unibe.ch/hpc), the HPC cluster at the University of Bern. The update to the systematic review was conducted with the help of Beatrice Minder from the library of the Institute of Social and Preventive Medicine at the University of Bern. 
 We also thank the staff and participants to the surveys that were included in this work.


\newpage

\bibliography{nnrti_res}
\bibliographystyle{unsrt}  


\end{document}
